\section*{Introduction} % Pas de numérotation
\addcontentsline{toc}{section}{Introduction} % Ajout dans la table des matières
  
La première année de master informatique à l’Université Paris 8 Vincennes-Saint-Denis se termine par un stage d’une durée minimum de 3 mois.
J'ai choisi de réaliser ce stage au sein du Laboratoire d’Études du Rayonnement et de la Matière en Astrophysique (LERMA) – Observatoire de Paris, situé à Paris 14ème. Les chercheurs de ce laboratoire travaillent avant tout à faire avancer le front des connaissances dans plusieurs axes de l'astronomie, notamment: la formation et l’évolution des galaxies, la formation de certaines étoiles, la chimie de la poussière interstellaire. Du logiciel libre est utilisé largement et est aussi produit: par exemple pour l'infrastructure de l'interféromètre ALMA et pour GDL destiné à être en libre accès pour tout le monde.
C’est ainsi que du 13 mai 2013 au 14 août 2013, je travaille sur un logiciel nommé GDL. Ce logiciel permet de faciliter le traitement des données. Ce projet utilise plusieurs bibliothèques open source, ce qui diminue le travail à réaliser et préserve l’utilisation de code testé pendant plusieurs années par une communauté ouverte.


Ma première approche sur ce projet a été de consacrer une première partie brève du stage à prendre en main les librairies utilisées tout en faisant les diverses installations qui me seront nécessaires pour la conception sur mon poste (GSL, Eigen, PLPlot, GraphicsMagick, …). Ensuite je prévoie une première ébauche de conception  me permettant mieux d’assimiler les bibliothèques, suivi d’une étude approfondie de besoin qui précède la conception spécifique de GDL.

Le travail effectué fut aussi réalisé à trois, avec mon tuteur de stage Alain Coulais et Mouadh Ayad, étudiant à l'Université Paris-Sud, IUT de Cachan et lui aussi stagiaire au LERMA travaillant sur un sujet complémentaire au mien, notamment l'élaboration d’une fonction SMOOTH et suite de tests de régression. En plus j'ai été en contact par courriel électronique avec Marc Schellens (fondateur et leader du projet GDL) et d'autres membres de la communauté. Le fait d’avoir travaillé en collaboration avec ces personnes a renforcé le cadre professionnel de ce stage car il est aujourd’hui impossible de travailler seul sur des projets de grande ou même petite envergure.


Dans la première partie de ce rapport je vais décrire le lieu du stage et l’environnement de travail. Après je vais présenter le cahier de charge et le but du stage. Dans la troisième et dernière partie je vais expliquer le travail que j’ai réalisé pendant ces trois mois, les difficultés que j’ai rencontrées, comment j’ai choisi de les résoudre et les résultats obtenus. Pour conclure le rapport, je résume les apports de ce stage, aussi bien personnels que professionnels.