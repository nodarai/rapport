\section*{Conclusion}
\addcontentsline{toc}{section}{Conclusion}

Le travail que j'ai réalisé lors du stage sur le projet GDL sera utilisé dans la prochain version du logiciel, qui va sortir à l'automne 2013. Ces corrections et améliorations vont aider GDL à devenir une alternative plus solide du logiciel propriétaire IDL (largement utilisé dans le domaine de l’astronomie). Tous les modifications sont mis dans le CVS et peuvent être téléchargées et compilées pour tester avant la sortie d'une version complète. Les utilisateurs ont déjà la possibilité de tester ces nouvelles fonctionnalités (les routines de Cholesky) et de ressentir la performance augmentée (grâce à la définition dynamique du nombre de threads pour OpenMP).\\

Ce stage au sein de l’Observatoire de Paris m'a été très profitable, et ce sur de nombreux plans. D'une part techniquement, j'ai pu découvrir un nouveau langage de programmation IDL/GDL, qui est un langage interprété. J'ai obtenu l’expérience de codage avec ce langage pendant que j'ai codé les différents tests de régression et de performance. Aussi j'ai approfondi mes connaissances en C++ durant plusieurs missions de correction de bugs et en ajoutant les nouvelles fonctionnalités dans GDL. Un autre apport de ce stage a été d'avoir utilisé plusieurs systèmes d'exploitation, qui m'a permis de me familiariser plus avec les différentes distributions de Linux (CentOS, Debian, Ubuntu). Ces systèmes étaient présents sur les serveurs aussi bien que sur les ordinateurs personnels.\\

Une des grandes difficultés techniques de ce stage fut d’avoir à modifier un code conçu et souvent modifié par d’autres personnes car il faut faire attention à bien comprendre pourquoi les développeurs précédents ont programmé de cette façon pour ensuite pouvoir apporter les modifications nécessaires. Mais le code qui n'est pas modifié pendent longtemps ne représente pas une logique triviale à comprendre non plus. En ce cas la difficulté est que l'auteur du code peut oublier la raison de choisir cette manière de codage. Dans tous les cas la difficulté du code augmente si le code n'est pas bien commenté et documenté. La documentation d'un logiciel est un aspect très important en informatique, car on a besoin d'utiliser les divers bibliothèques ou juste des petits morceaux du code de l'autre logiciel. Une bonne documentation facilite largement la vie des développeurs.\\  \\ \\

Les apports personnelles de ce stage sont tout aussi nombreux. Sur le plan professionnel, ce stage m’a apporté de nouvelles connaissances et a augmenté mes capacités de compréhension. Sur le plan humain, l’enrichissement est incontestable puisque j’ai pu développer mon indépendance et travailler en toute autonomie. J’ai par ailleurs pu juger de l’importance du relationnel entre les différents services, mais aussi entre les personnes internes à un service. Cette importance a dévoilé pendent les problèmes techniques (par exemple un problème avec la cable VGA), auxquelles la réaction a été très rapide et tout le monde a montré sa volonté d'aider.\\

Enfin, la communication fut un point important lors de ce stage. Le travail en équipe fait partie de la vie d’un ingénieur, ainsi, j’ai beaucoup appris de ce stage grâce au fait d’avoir travailler dans un laboratoire en collaboration avec un autre stagiaire. Aussi, le besoin de faire
des réunions pour faire le point sur l’avancement du projet s’est très vite fait ressentir afin de vérifier que les idées convergeaient et pour décider des suites à donner aux activités.

