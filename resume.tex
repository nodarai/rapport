\section*{Résumé}
\addcontentsline{toc}{section}{Résumé} 

Mon stage s’est déroulé au sein du Laboratoire d'Etudes du Rayonnement et de la Matière en Astrophysique (LERMA) – Observatoire de Paris. Son but a été de contribuer au développement de GDL (clone libre d'IDL - logiciel largement utilisé en astronomie professionnelle) pour atteindre un niveau de maturité et de facilité d'utilisation garantissant la préservation à long terme des capacités de traitements et d'analyse pour de nombreuses expériences en astronomie au sol ou spatiale.

Ce stage était principalement destiné à la mise en place de:

1) fonctionnalités manquantes,

2) correction des bugs connus,

3) tests de régression et performance.


Ce stage nécessitait une connaissance solide du langage C++, car le projet GDL contient plus de 120 000 lignes écrites en ce langage. J'ai aussi eu à apprendre la syntaxe IDL/GDL pour pouvoir écrire des tests qui vont assurer la qualité des nouvelles fonctionnalités.


\section*{Summary}

I've done my internship at LERMA - Observatoire de Paris. The main goal was to contribute to GDL (free clone of IDL - very popular software in the domain of astronomy) to reach a level of maturity and usability ensuring long term preservation of analysis capabilities for numerous ground experiments and spaces missions based on IDL.

The course of internship was to implement:

1) missing functionalities,

2) correction of existing bugs,

3) performance and regression tests.

Solid knoledge of C++ language was necessary for this internship, beacause the GDL project contains more than 120 000 lines of code in that programming language. I've learned IDL/GDL syntaxe as well, that I've used to code the tests for ensuring the quality of recently implemented fonctionalities.
